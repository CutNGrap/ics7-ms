\usepackage[utf8]{inputenc}
\usepackage[T2A]{fontenc}
\usepackage[english,russian]{babel}
\usepackage{cmap}
\usepackage{enumitem}

\usepackage{csvsimple}
\usepackage{hyphenat} 
\usepackage{amstext, amsmath,amsfonts,amssymb,amsthm,mathtools} 

\usepackage{setspace}
\onehalfspacing 

\usepackage{geometry}
\geometry{left=30mm}
\geometry{right=10mm}
\geometry{top=10mm}
\geometry{bottom=20mm}

\usepackage{graphicx}
\graphicspath{{img/}} 

\usepackage{threeparttable}
\usepackage{bigdelim}

\usepackage{setspace}
\onehalfspacing % Полуторный интервал

\frenchspacing
\usepackage{indentfirst} % Красная строка

\usepackage{cases}

\usepackage[unicode,pdftex]{hyperref} % Ссылки в pdf
\hypersetup{hidelinks}

\usepackage{xcolor}
\usepackage{listings}
% Для листинга кода:
\lstset{%
	language=Matlab,   					% выбор языка для подсветки	
	basicstyle=\small\sffamily,			% размер и начертание шрифта для подсветки кода
	numbers=left,						% где поставить нумерацию строк (слева\справа)
	%numberstyle=,					% размер шрифта для номеров строк
	stepnumber=1,						% размер шага между двумя номерами строк
	numbersep=5pt,						% как далеко отстоят номера строк от подсвечиваемого кода
	frame=single,						% рисовать рамку вокруг кода
	tabsize=4,							% размер табуляции по умолчанию равен 4 пробелам
	captionpos=t,						% позиция заголовка вверху [t] или внизу [b]
	breaklines=true,					
	breakatwhitespace=true,				% переносить строки только если есть пробел
	escapeinside={\#*}{*)},				% если нужно добавить комментарии в коде
	backgroundcolor=\color{white},
}
\usepackage{caption}
\captionsetup{labelsep = endash}
\captionsetup[figure]{name = {}, justification = centerlast,labelformat=empty}
\captionsetup[lstlisting]{justification = justified}
\captionsetup[table]{justification = justified}


\usepackage{titlesec}
\titleformat{\chapter}{\LARGE\bfseries}{\thechapter}{20pt}{\LARGE\bfseries}
\titleformat{\section}{\Large\bfseries}{\thesection}{20pt}{\Large\bfseries}

\renewcommand\labelitemi{\ --\ }
\renewcommand\labelenumi{\theenumi)}

\newcommand{\img}[3] {
	\begin{figure}[h!]
		\includegraphics[scale = #1]{img/#2}
		\caption{#3}
		\label{img:#2}
	\end{figure}
}

\newcommand{\lst}[4]{
\lstinputlisting[language=Matlab, firstline=#1, lastline=#2, label=lst:#3,caption=#4]{../src/code.m}
}

\usepackage{siunitx,array,booktabs}
\usepackage{amsthm}
\usepackage{tabularx}
