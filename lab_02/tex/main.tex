% !TeX spellcheck = ru_RU
\documentclass[a4paper, 14pt, unknownkeysallowed]{extreport}
\usepackage[utf8]{inputenc}
\usepackage[T2A]{fontenc}
\usepackage[english,russian]{babel}
\usepackage{cmap}
\usepackage{enumitem}

\usepackage{csvsimple}
\usepackage{hyphenat} 
\usepackage{amstext, amsmath,amsfonts,amssymb,amsthm,mathtools} 

\usepackage{setspace}
\onehalfspacing 

\usepackage{geometry}
\geometry{left=30mm}
\geometry{right=10mm}
\geometry{top=10mm}
\geometry{bottom=20mm}

\usepackage{graphicx}
\graphicspath{{img/}} 

\usepackage{threeparttable}
\usepackage{bigdelim}

\usepackage{setspace}
\onehalfspacing % Полуторный интервал

\frenchspacing
\usepackage{indentfirst} % Красная строка

\usepackage{cases}

\usepackage[unicode,pdftex]{hyperref} % Ссылки в pdf
\hypersetup{hidelinks}

\usepackage{xcolor}
\usepackage{listings}
% Для листинга кода:
\lstset{%
	language=Matlab,   					% выбор языка для подсветки	
	basicstyle=\small\sffamily,			% размер и начертание шрифта для подсветки кода
	numbers=left,						% где поставить нумерацию строк (слева\справа)
	%numberstyle=,					% размер шрифта для номеров строк
	stepnumber=1,						% размер шага между двумя номерами строк
	numbersep=5pt,						% как далеко отстоят номера строк от подсвечиваемого кода
	frame=single,						% рисовать рамку вокруг кода
	tabsize=4,							% размер табуляции по умолчанию равен 4 пробелам
	captionpos=t,						% позиция заголовка вверху [t] или внизу [b]
	breaklines=true,					
	breakatwhitespace=true,				% переносить строки только если есть пробел
	escapeinside={\#*}{*)},				% если нужно добавить комментарии в коде
	backgroundcolor=\color{white},
}
\usepackage{caption}
\captionsetup{labelsep = endash}
\captionsetup[figure]{name = {Рисунок}, justification = centerlast}
\captionsetup[lstlisting]{justification = justified}
\captionsetup[table]{justification = justified}

\usepackage{titlesec}
\titleformat{\chapter}{\LARGE\bfseries}{\thechapter}{20pt}{\LARGE\bfseries}
\titleformat{\section}{\Large\bfseries}{\thesection}{20pt}{\Large\bfseries}

\renewcommand\labelitemi{\ --\ }
\renewcommand\labelenumi{\theenumi)}

\newcommand{\img}[3] {
	\begin{center}
		\begin{figure}[h!]
			\includegraphics[scale = #1]{img/#2}
			\caption{#3}
			\label{img:#2}
		\end{figure}
	\end{center}
}

\newcommand{\lst}[4]{
\lstinputlisting[language=Matlab, firstline=#1, lastline=#2, label=lst:#3,caption=#4]{../src/code.m}
}

\usepackage{siunitx,array,booktabs}
\usepackage{amsthm}
\usepackage{tabularx}
\newtheorem{theorem}{Theorem}
\newtheorem{definition}{Опредление}


\begin{document}
\include{title}
\setcounter{page}{2}

\chapter{Постановка задачи}
Цель работы: построение доверительных интервалов для математического ожидания и дисперсии нормальной случайной величины.
\section{Содержание работы}
\begin{enumerate}
\item Для выборки объема n из нормальной генеральной совокупности X реализовать в виде программы на ЭВМ
	\begin{enumerate}
		\item вычисление точечных оценок $\hat{\mu}(\vec{x}_n)$ и $S^2(\vec{x}_n)$
		MX и дисперсии DX; соответственно;
		\item вычисление нижней и верхней границ $\underline{{\mu}}(\vec{x}_n)$, $\overline{{\mu}}(\vec{x}_n)$ для $\gamma$-доверительного интервала для
		математического ожидания MX;
		\item вычисление нижней и верхней границ 
		$\underline{\sigma^2}(\vec{x}_n)$, 
		$\overline{\sigma^2}(\vec{x}_n)$ для $\gamma$-доверительного интервала для
		дисперсии DX;
	\end{enumerate}
	\item вычислить $\hat{\mu}(\vec{x}_n)$ и $S^2(\vec{x}_n)$ для выборки из индивидуального варианта;
	\item для заданного пользователем уровня доверия $\gamma$ и N – объема выборки из индивидуального варианта:
	\begin{enumerate}
		\item на координатной плоскости $Oyn$ построить прямую $y=\hat{\mu}(\vec{x}_N)$, также графики функций $y=\hat{\mu}(\vec{x}_n)$
			$y=\underline{{\mu}}(\vec{x}_n)$, $y=\overline{{\mu}}(\vec{x}_n)$,  как функций объема n выборки, где n изменяется от 1 до N;
		\item на другой координатной плоскости $Ozn$ построить прямую $z = S^2(\vec{x}_N)$, также графики
			функций $z = S^2(\vec{x}_n)$, $z = \underline{\sigma^2}(\vec{x}_n)$
			и $z = \overline{\sigma^2}(\vec{x}_n)$ как функций объема n выборки, где n
			изменяется от 1 до N.
	\end{enumerate}
\end{enumerate}

\chapter{Теоретическая часть}
\addcontentsline{toc}{chapter}{Введение}

Пусть $\vec{x} = (x_1, \dots, x_n)$ --- реализация случайной выборки из генеральной совокупности случайно величины X, закон распределения которой известен с точностью до параметра $\theta$.

$\gamma$--доверительным интервалом для параметра $\theta$ называется интервал $(\underline{\theta}, \overline{\theta})$ для которого справедливо $P(\underline{\theta} \leq \theta \leq \overline{\theta}) = \gamma$.

В работе использовались следующие формулы для вычисления величин:
\begin{gather}
	\label{mu}
	\hat\mu = \frac{1}{n}\sum_{i = 1}^{N}~x_i;\\
	\label{s}
	S^2 = \frac{1}{n - 1}\sum_{i = 1}^{N}~(x_i - \hat\mu)^2;\\
	\label{mu_low}
	\underline{\mu}(\vec{x}_n) = \overline{X} - \frac{S}{\sqrt{n}}t_{1-\alpha}(n-1)\\
	\label{mu_high}
	\overline{\mu}(\vec{x}_n) = \overline{X} + \frac{S}{\sqrt{n}}t_{1-\alpha}(n-1)\\
	\label{sigma_low}
	\underline{\sigma}(\vec{x}_n) = \frac{S^2(n-1)}{\chi^2_{1-\alpha}(n-1)}\\
	\label{sigma_high}
	\overline{\sigma}(\vec{x}_n) = \frac{S^2(n-1)}{\chi^2_\alpha(n-1)}
\end{gather}

где (\ref{mu}) --- точечная оценка математического ожидания, (\ref{s}) --- точечная оценка дисперсии, (\ref{mu_low}) --- нижняя граница $\gamma$--доверительного интервала для математического ожидания, (\ref{mu_high}) --- верхняя граница $\gamma$--доверительного интервала для математического ожидания, (\ref{sigma_low}) --- нижняя граница $\gamma$--довери-тельного интервала для дисперсии, верхняя граница $\gamma$--доверительного интервала для дисперсии, $t_{\alpha}(n-1)$ --- квантиль уровня $(1 - \alpha)$ распределения Стьюдента c (n - 1) степенями свободы, $\chi^2_{\alpha}(n-1)$ --- квантиль уровня $\alpha$ распределения $\chi^2$.

\chapter{Практическая часть}
\section{Результаты расчетов для выборки из индивидуального варианта}

Значения параметров для выборки из индивидуального варианта №2:

$\hat{\mu}(\overrightarrow{x}_n)= -0.285917$;

$S^2(\overrightarrow{x}_n)= 0.917021$.

Результаты построения графиков функций приведены на рисунках \ref{img:1}, \ref{img:2}.

\img{0.7}{1}{Графики прямой $y = \hat{\mu}(\vec{x}_N)$, а также функций $y = \hat{\mu}(\vec{x}_n)$
	$y=\underline{{\mu}}(\vec{x}_n)$, $y=\overline{{\mu}}(\vec{x}_n)$,  как функций объема n выборки, где n изменяется от 1 до N}
\img{0.7}{2}{Графики прямой $z = S^2(\vec{x}_N)$, также
	функций $z = S^2(\vec{x}_n)$, $z = \underline{\sigma^2}(\vec{x}_n)$
	и $z = \overline{\sigma^2}(\vec{x}_n)$ как функций объема n выборки, где n
	изменяется от 1 до N}
\end{document}