% !TeX spellcheck = ru_RU
\documentclass[a4paper, 14pt, unknownkeysallowed]{extreport}
\usepackage[utf8]{inputenc}
\usepackage[T2A]{fontenc}
\usepackage[english,russian]{babel}
\usepackage{cmap}
\usepackage{enumitem}

\usepackage{csvsimple}
\usepackage{hyphenat} 
\usepackage{amstext, amsmath,amsfonts,amssymb,amsthm,mathtools} 

\usepackage{setspace}
\onehalfspacing 

\usepackage{geometry}
\geometry{left=30mm}
\geometry{right=10mm}
\geometry{top=10mm}
\geometry{bottom=20mm}

\usepackage{graphicx}
\graphicspath{{img/}} 

\usepackage{threeparttable}
\usepackage{bigdelim}

\usepackage{setspace}
\onehalfspacing % Полуторный интервал

\frenchspacing
\usepackage{indentfirst} % Красная строка

\usepackage{cases}

\usepackage[unicode,pdftex]{hyperref} % Ссылки в pdf
\hypersetup{hidelinks}

\usepackage{xcolor}
\usepackage{listings}
% Для листинга кода:
\lstset{%
	language=Matlab,   					% выбор языка для подсветки	
	basicstyle=\small\sffamily,			% размер и начертание шрифта для подсветки кода
	numbers=left,						% где поставить нумерацию строк (слева\справа)
	%numberstyle=,					% размер шрифта для номеров строк
	stepnumber=1,						% размер шага между двумя номерами строк
	numbersep=5pt,						% как далеко отстоят номера строк от подсвечиваемого кода
	frame=single,						% рисовать рамку вокруг кода
	tabsize=4,							% размер табуляции по умолчанию равен 4 пробелам
	captionpos=t,						% позиция заголовка вверху [t] или внизу [b]
	breaklines=true,					
	breakatwhitespace=true,				% переносить строки только если есть пробел
	escapeinside={\#*}{*)},				% если нужно добавить комментарии в коде
	backgroundcolor=\color{white},
}
\usepackage{caption}
\captionsetup{labelsep = endash}
\captionsetup[figure]{name = {Рисунок}, justification = centerlast}
\captionsetup[lstlisting]{justification = justified}
\captionsetup[table]{justification = justified}

\usepackage{titlesec}
\titleformat{\chapter}{\LARGE\bfseries}{\thechapter}{20pt}{\LARGE\bfseries}
\titleformat{\section}{\Large\bfseries}{\thesection}{20pt}{\Large\bfseries}

\renewcommand\labelitemi{\ --\ }
\renewcommand\labelenumi{\theenumi)}

\newcommand{\img}[3] {
	\begin{center}
		\begin{figure}[h!]
			\includegraphics[scale = #1]{img/#2}
			\caption{#3}
			\label{img:#2}
		\end{figure}
	\end{center}
}

\newcommand{\lst}[4]{
\lstinputlisting[language=Matlab, firstline=#1, lastline=#2, label=lst:#3,caption=#4]{../src/code.m}
}

\usepackage{siunitx,array,booktabs}
\usepackage{amsthm}
\usepackage{tabularx}
\newtheorem{theorem}{Theorem}
\newtheorem{definition}{Опредление}


\begin{document}
\include{title}
\setcounter{page}{2}

\chapter{Постановка задачи}
Цель работы: построение гистограммы и эмпирической функции распределения.
\section{Содержание работы}
\begin{enumerate}
	\item Для выборки объема n из генеральной совокупности X реализовать в виде программы на ЭВМ
	\begin{enumerate}
		\item вычисление максимального значения Mmax и минимального значения Mmin;
		\item размаха R выборки;
		\item вычисление оценок $\hat{\mu}$ и $S^2$ математического ожидания MX и дисперсии DX;
		\item группировку значений выборки в m = [log2 n] + 2 интервала;
		\item построение на одной координатной плоскости гистограммы и графика функции плотности распределения вероятностей нормальной случайной величины с математическим
		ожиданием  $\hat{\mu}$ и дисперсией $S^2$;
		\item построение на другой координатной плоскости графика эмпирической функции распределения и функции распределения нормальной случайной величины с математическим
		ожиданием $\hat{\mu}$ и дисперсией $S^2$.
	\end{enumerate}
	\item Провести вычисления и построить графики для выборки из индивидуального варианта.
\end{enumerate}

\chapter{Теоретическая часть}
\addcontentsline{toc}{chapter}{Введение}

\section{Формулы для вычисления величин}
Реализация случайной выборки:
\begin{equation}
	\label{x}
	\overrightarrow{x} = (x_1, x_2,\dots,x_n).
\end{equation}
Минимальное значение выборки:
\begin{equation}
	\label{min}
	M_{min} = min(x_1, x_2,\dots,x_n).
\end{equation}
Максимальное значение выборки:
\begin{equation}
	\label{max}
	M_{max} = max(x_1, x_2,\dots,x_n).
\end{equation}
Размах выборки:
\begin{equation}
	\label{r}
	R = M_{max} - M_{min}.
\end{equation}
Оценка математического ожидания выборки:
\begin{equation}
	\label{mu}
	\hat{\mu}(\overrightarrow{x}) = \overline{x} = \frac{1}{n}\sum_{i=1}^{n}x_i.
\end{equation}
Несмещенная(исправленная) оценка дисперсии выборки:
\begin{equation}
	\label{s2}
	S^2(\overrightarrow{x}) = \frac{1}{n-1}\sum_{i=1}^{n}{(x_i - \overline{x})^2}.
\end{equation}

\section{Эмпирическая плотность и гистограмма}
\begin{definition}
	Эмпирической функцией плотности, отвечающей выборке  $\overline{x}$, называют функцию
	\begin{equation}
		f_{n} (x) = \begin{cases}
			\frac{m_i}{n\delta}, \text{если $x\in J_i$},\\
			0, \text{иначе},
		\end{cases}
	\end{equation}
\end{definition}
	где 
	\begin{itemize}
		\item m --- количество полуинтервалов интервала J = [$M_{min}$, $M_{max}$], $\overline{x}$, которые приняли значение меньше x,
		\item $m_i$ --- количество элементов выборки  $\overline{x}$, принадлежащих полуинтервалу $J_i$, $i=\overline{1,m}$,
		\item n --- количество элементов выборки  $\overline{x}$,
		\item $\delta = \frac{M_{min} - M_{max}}{m} = \frac{|J|}{m}$.
	\end{itemize}

\begin{definition}
	График функции $f_{n}(x)$ называют гистограммой.
\end{definition}

\section{Эмпирическая функция распределения}
\begin{definition}
	Эмпирической функцией распределения, отвечающей выборке $\overline{x}$, называют функцию\\
	\begin{equation}
	F_n(x, \overline{x}) = \frac{n(x,\overline{x})}{n}, x\in\mathbb{R},
	\end{equation}
\end{definition}
	где
	\begin{itemize}
	\item $n(x,\overline{x})$ --- количество элементов выборки $\overline{x}$, которые приняли значение меньше x,
	\item n --- количество элементов выборки.
	\end{itemize}

\chapter{Практическая часть}
\section{Текст программы}
\lst{1}{68}{1}{Текст программы}


\clearpage
\section{Результаты расчетов для выборки из индивидуального варианта}

Задание выполнялось по варианту №2.  
\begin{center}
\begin{table}[h!]
	\caption{Значения параметров для выборки из индивидуального варианта}
\begin{tabularx}{\textwidth} { 
		| >{\centering\arraybackslash}X 
		| >{\centering\arraybackslash}X| }
	\hline
	$M_{min}$ & -2.79 \\
	\hline
	$M_{max}$  & 1.8  \\
	\hline
	$R$  & 4.59  \\\hline
	$\hat{\mu}(\overrightarrow{x})$  & -0.285917  \\\hline
	$S^2(\overrightarrow{x})$  & 0.917021  \\
	\hline
	$m$  & 8  \\
	\hline
	$\delta$  & 0.57375  \\
	\hline
\end{tabularx}
\end{table}

	
\begin{table}[h!]
	\caption{Интервальная группировка значений выборки при m = 8}
\begin{tabularx}{\textwidth} {
		| >{\centering\arraybackslash}X 
		| >{\centering\arraybackslash}X
		| >{\centering\arraybackslash}X
		| >{\centering\arraybackslash}X|}
	\hline
	[-2.79, -2.22) &[-2.22, -1.64)& [-1.64, -1.07) &[-1.07, -0.50) \\
	\hline
	5  & 5 & 15 &18\\
	\hline
	[-0.50,  0.08) &[ 0.08,  0.65) &[ 0.65,  1.23) &[ 1.23,  1.80]\\
	\hline
	35& 24& 12 &6\\
	\hline
\end{tabularx}
\end{table}
\end{center}
\img{0.7}{ex}{Результат работы программы}
Результаты построения гистограммы и графика функции плотности распределения вероятностей нормальной случайной величины с математическим ожиданием  $\hat{\mu}$ и дисперсией $S^2$, а также 
построения графика эмпирической функции распределения и функции распределения нормальной случайной величины с математическим ожиданием $\hat{\mu}$ и дисперсией $S^2$ приведены на рисунках \ref{img:1}, \ref{img:2}.
\img{0.87}{1}{Гистограмма и график функции плотности распределения вероятностей нормальной случайной величины с математическим
	ожиданием  $\hat{\mu}$ и дисперсией $S^2$}
\img{0.87}{2}{График эмпирической функции распределения и функции распределения нормальной случайной величины с математическим
	ожиданием $\hat{\mu}$ и дисперсией $S^2$}
\end{document}